%%%%%%%%%%%%%%%%%%%%%%%%%%%%%%%%%%%%%%%%%%%%%%%%%%%%%%%%%%%%%%%%%%%%%%%%%%%%%%%%%%%%%%%%%%%%%
%%%%%%%%%%%%%%%%%%                Realizator: Pop Adrian              %%%%%%%%%%%%%%%%%%%%%%%
%%%%%%%%%%%%%%%%%%%%%%%%%%%%%%%%%%%%%%%%%%%%%%%%%%%%%%%%%%%%%%%%%%%%%%%%%%%%%%%%%%%%%%%%%%%%%

\makeatletter

\pgf@circ@Rlen = \pgfkeysvalueof{/tikz/circuitikz/bipoles/length}
\def\TikzBipolePath#1#2{\pgf@circ@bipole@path{#1}{#2}}
\makeatother

\newlength{\ResUp} 
\newlength{\ResRight}

%%%%%%%%%%%%%%%%%%%%%%%%%%%%%%%%%%%%%%%%%%%%%%%%%%%%%%%%%%%%%%%%%%%%%%%%%%%%%%%%%%%%%%%%%%%%%
%%%%%%%%%%%%%%%%%%      Sursa de curent: simbol folosit in Romania    %%%%%%%%%%%%%%%%%%%%%%%
%%%%%%%%%%%%%%%%%%%%%%%%%%%%%%%%%%%%%%%%%%%%%%%%%%%%%%%%%%%%%%%%%%%%%%%%%%%%%%%%%%%%%%%%%%%%%
%Declaram dimensiunile initiale
\ctikzset{bipoles/romanianCurrentSource/height/.initial=.60}
\ctikzset{bipoles/romanianCurrentSource/width/.initial=.60}

%Definim noul simbol pentru SIC
\pgfcircdeclarebipole{} 
	%Offset pentru label-uri
	{\ctikzvalof{bipoles/romanianCurrentSource/height}}
	%Numele simbolului
	{romanianCurrentSource}
	%Dimensiunile "cutiei" in care va sta
	{\ctikzvalof{bipoles/romanianCurrentSource/height}}
	{\ctikzvalof{bipoles/romanianCurrentSource/width}}
	{
		%Stabilim grosimea standard a liniei pentru un element
		\pgfsetlinewidth{\pgfkeysvalueof{/tikz/circuitikz/bipoles/thickness}\pgfstartlinewidth}
		
		%Definim ancorele
		\pgfextracty{\ResUp}{\northeast}
		\pgfextractx{\ResRight}{\southwest}
		
		%Desenam cerculetul
		\pgfpathellipse{\pgfpointorigin}{\pgfpoint{0cm}{\ResUp}}{\pgfpoint{\ResRight}{0cm}}
		
		%Desenam prima sagetica; trebuie sa avem grija ca dupa ce desenam un segment,
		%urmatoarea linie va fi desenata relativ la capatul segmentului, nu fata de
		%origine, deci sunt necesare cateva repozitionari
		\pgfmoveto{\pgfpoint{1.0\ResRight}{0.0\ResUp}}   %Ne pozitionam in (-1, 0)
		\pgflineto{\pgfpoint{0.1\ResRight}{0.0\ResUp}}   %Desenam corpul sagetii
		\pgflineto{\pgfpoint{0.3\ResRight}{-0.25\ResUp}} %Desenam unul din capete
		\pgfmoveto{\pgfpoint{0.1\ResRight}{0.0\ResUp}}   %Ne pozitionam in (-0.1, 0)
	    \pgflineto{\pgfpoint{0.3\ResRight}{0.25\ResUp}}  %Desenam celalalt capat
		
		%Desenam ce-a de-a doua sagetica
		\pgfmoveto{\pgfpoint{-0.2\ResRight}{0.0\ResUp}}  %Ne pozitionam in (0.2, 0)
		\pgflineto{\pgfpoint{-1.0\ResRight}{0.0\ResUp}}  %Desenam corpul sagetii
		\pgfmoveto{\pgfpoint{0.0\ResRight}{0.25\ResUp}}  %Ne repozitionam
		\pgflineto{\pgfpoint{-0.2\ResRight}{0.0\ResUp}}  %Desenam unul din capete
		\pgflineto{\pgfpoint{0.0\ResRight}{-0.25\ResUp}} %Desenam celalat capat
		
		%Pentru desenare, sa folosim functia draw
		\pgfusepath{draw}
	}

%Ii definim un stil si o cale si...gata!
\def\romanianCurrentSourcepath#1{\TikzBipolePath{romanianCurrentSource}{#1}}
\tikzset{romanianCurrentSource/.style = {\circuitikzbasekey, /tikz/to path=\romanianCurrentSourcepath, l=#1}}


%%%%%%%%%%%%%%%%%%%%%%%%%%%%%%%%%%%%%%%%%%%%%%%%%%%%%%%%%%%%%%%%%%%%%%%%%%%%%%%%%%%%%%%%%%%%%
%%%%%%%%%%%%%%%%%%      Sursa de tensiune: simbol folosit in Romania    %%%%%%%%%%%%%%%%%%%%
%%%%%%%%%%%%%%%%%%%%%%%%%%%%%%%%%%%%%%%%%%%%%%%%%%%%%%%%%%%%%%%%%%%%%%%%%%%%%%%%%%%%%%%%%%%%%
%Declaram dimensiunile initiale
\ctikzset{bipoles/romanianVoltageSource/height/.initial=.60}
\ctikzset{bipoles/romanianVoltageSource/width/.initial=.60}

%Definim noul simbol pentru SIT
\pgfcircdeclarebipole{}
	%Stabilim grosimea standard a liniei pentru un element
	{\ctikzvalof{bipoles/romanianVoltageSource/height}}
	%Numele simbolului
	{romanianVoltageSource}
	%Dimensiunile "cutiei" in care va sta
	{\ctikzvalof{bipoles/romanianVoltageSource/height}}
	{\ctikzvalof{bipoles/romanianVoltageSource/width}}
	{
		%Stabilim grosimea standard a liniei pentru un element
		\pgfsetlinewidth{\pgfkeysvalueof{/tikz/circuitikz/bipoles/thickness}\pgfstartlinewidth}
		
		%Definim ancorele
		\pgfextracty{\ResUp}{\northeast}
		\pgfextractx{\ResRight}{\southwest}
		
		%Desenam cerculetul
		\pgfpathellipse{\pgfpointorigin}{\pgfpoint{0}{\ResUp}}{\pgfpoint{\ResRight}{0}}
		
		%Desenam prima sagetica; trebuie sa avem grija ca dupa ce desenam un segment,
		%urmatoarea linie va fi desenata relativ la capatul segmentului, nu fata de
		%origine, deci sunt necesare cateva repozitionari
		\pgfmoveto{\pgfpoint{1.0\ResRight}{0.0\ResUp}}   %Ne pozitionam in (-1, 0)
		\pgflineto{\pgfpoint{-1.0\ResRight}{0.0\ResUp}}   %Desenam corpul sagetii
		\pgflineto{\pgfpoint{-0.7\ResRight}{-0.25\ResUp}} %Desenam unul din capete
		\pgfmoveto{\pgfpoint{-1.0\ResRight}{0.0\ResUp}}   %Ne pozitionam in (-0.1, 0)
	    \pgflineto{\pgfpoint{-0.7\ResRight}{0.25\ResUp}}  %Desenam celalalt capat
		
		%Pentru desenare, sa folosim functia draw
		\pgfusepath{draw}
	}
	\def\romanianVoltageSourcepath#1{\TikzBipolePath{romanianVoltageSource}{#1}}
%Ii stabilim un nume, un posibil label si...gata!
\tikzset{romanianVoltageSource/.style = {\circuitikzbasekey, /tikz/to path=\romanianVoltageSourcepath, l=#1}}


%%%%%%%%%%%%%%%%%%%%%%%%%%%%%%%%%%%%%%%%%%%%%%%%%%%%%%%%%%%%%%%%%%%%%%%%%%%%%%%%%%%%%%%%%%%%%
%%%%%%%%%%%%%%      Sursa comandata de curent: simbol folosit in Romania    %%%%%%%%%%%%%%%%%
%%%%%%%%%%%%%%%%%%%%%%%%%%%%%%%%%%%%%%%%%%%%%%%%%%%%%%%%%%%%%%%%%%%%%%%%%%%%%%%%%%%%%%%%%%%%%
%Declaram dimensiunile initiale
\ctikzset{bipoles/romanianCCS/height/.initial=.60}
\ctikzset{bipoles/romanianCCS/width/.initial=.60}

%Definim noul simbol pentru SCC
\pgfcircdeclarebipole{} 
	%Offset pentru label-uri
	{\ctikzvalof{bipoles/romanianCCS/height}}
	%Numele simbolului
	{romanianCCS}
	%Dimensiunile "cutiei" in care va sta
	{\ctikzvalof{bipoles/romanianCCS/height}}
	{\ctikzvalof{bipoles/romanianCCS/width}}
	{
		%Stabilim grosimea standard a liniei pentru un element
		\pgfsetlinewidth{\pgfkeysvalueof{/tikz/circuitikz/bipoles/thickness}\pgfstartlinewidth}
		
		%Definim ancorele
		\pgfextracty{\ResUp}{\northeast}
		\pgfextractx{\ResRight}{\southwest}
		
		%Desenam rombul
		\pgftransformrotate{-45}
		\pgfpathrectanglecorners{\southwest}{\northeast}
		\pgftransformrotate{45}

		%Desenam prima sagetica; trebuie sa avem grija ca dupa ce desenam un segment,
		%urmatoarea linie va fi desenata relativ la capatul segmentului, nu fata de
		%origine, deci sunt necesare cateva repozitionari
		\pgfmoveto{\pgfpoint{1.25\ResRight}{0.0\ResUp}}   %Ne pozitionam in (-1, 0)
		\pgflineto{\pgfpoint{0.1\ResRight}{0.0\ResUp}}   %Desenam corpul sagetii
		\pgflineto{\pgfpoint{0.3\ResRight}{-0.25\ResUp}} %Desenam unul din capete
		\pgfmoveto{\pgfpoint{0.1\ResRight}{0.0\ResUp}}   %Ne pozitionam in (-0.1, 0)
	    \pgflineto{\pgfpoint{0.3\ResRight}{0.25\ResUp}}  %Desenam celalalt capat
		
		%Desenam ce-a de-a doua sagetica
		\pgfmoveto{\pgfpoint{-0.2\ResRight}{0.0\ResUp}}  %Ne pozitionam in (0.2, 0)
		\pgflineto{\pgfpoint{-1.25\ResRight}{0.0\ResUp}}  %Desenam corpul sagetii
		\pgfmoveto{\pgfpoint{0.0\ResRight}{0.25\ResUp}}  %Ne repozitionam
		\pgflineto{\pgfpoint{-0.2\ResRight}{0.0\ResUp}}  %Desenam unul din capete
		\pgflineto{\pgfpoint{0.0\ResRight}{-0.25\ResUp}} %Desenam celalat capat
		
		%Pentru desenare, sa folosim functia draw
		\pgfusepath{draw}
	}

%Ii definim un stil si o cale si...gata!
\def\romanianCCS#1{\TikzBipolePath{romanianCCS}{#1}}
\tikzset{romanianCCS/.style = {\circuitikzbasekey, /tikz/to path=\romanianCCS, l=#1}}


%%%%%%%%%%%%%%%%%%%%%%%%%%%%%%%%%%%%%%%%%%%%%%%%%%%%%%%%%%%%%%%%%%%%%%%%%%%%%%%%%%%%%%%%%%%%%
%%%%%%%%%%%%%      Sursa comandata de tensiune: simbol folosit in Romania    %%%%%%%%%%%%%%%%
%%%%%%%%%%%%%%%%%%%%%%%%%%%%%%%%%%%%%%%%%%%%%%%%%%%%%%%%%%%%%%%%%%%%%%%%%%%%%%%%%%%%%%%%%%%%%
%Declaram dimensiunile initiale
\ctikzset{bipoles/romanianCVS/height/.initial=.60}
\ctikzset{bipoles/romanianCVS/width/.initial=.60}

%Definim noul simbol pentru CVS
\pgfcircdeclarebipole{}
	%Stabilim grosimea standard a liniei pentru un element
	{\ctikzvalof{bipoles/romanianCVS/height}}
	%Numele simbolului
	{romanianCVS}
	%Dimensiunile "cutiei" in care va sta
	{\ctikzvalof{bipoles/romanianCVS/height}}
	{\ctikzvalof{bipoles/romanianCVS/width}}
	{
		%Stabilim grosimea standard a liniei pentru un element
		\pgfsetlinewidth{\pgfkeysvalueof{/tikz/circuitikz/bipoles/thickness}\pgfstartlinewidth}
		
		%Definim ancorele
		\pgfextracty{\ResUp}{\northeast}
		\pgfextractx{\ResRight}{\southwest}
		
		%Desenam rombul
		\pgftransformrotate{-45}
		\pgfpathrectanglecorners{\southwest}{\northeast}
		\pgftransformrotate{45}
		
		%Desenam prima sagetica; trebuie sa avem grija ca dupa ce desenam un segment,
		%urmatoarea linie va fi desenata relativ la capatul segmentului, nu fata de
		%origine, deci sunt necesare cateva repozitionari
		\pgfmoveto{\pgfpoint{1.25\ResRight}{0.0\ResUp}}   %Ne pozitionam in (-1.25, 0)
		\pgflineto{\pgfpoint{-1.3\ResRight}{0.0\ResUp}}  %Desenam corpul sagetii
		\pgflineto{\pgfpoint{-0.9\ResRight}{-0.25\ResUp}} %Desenam unul din capete
		\pgfmoveto{\pgfpoint{-1.3\ResRight}{0.0\ResUp}}   %Ne pozitionam in (-0.1, 0)
	    \pgflineto{\pgfpoint{-0.9\ResRight}{0.25\ResUp}}  %Desenam celalalt capat
		
		%Pentru desenare, sa folosim functia draw
		\pgfusepath{draw}
	}
	\def\romanianCVS#1{\TikzBipolePath{romanianCVS}{#1}}
%Ii stabilim un nume, un posibil label si...gata!
\tikzset{romanianCVS/.style = {\circuitikzbasekey, /tikz/to path=\romanianCVS, l=#1}}